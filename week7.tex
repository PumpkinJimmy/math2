\documentclass[UTF8]{ctexart}
\usepackage{amsmath}
\title{第七周作业}
\author{邓贤杰}
\date{\today}
\begin{document}
    \maketitle
    \paragraph*{1}
    \subparagraph*{(1)}
    \begin{align*}
        \lambda^2-3\lambda+2 &= 0 \\
        (\lambda -1)(\lambda - 2) &= 0 \\
        y &= C_1 e^x + c_2 e^{2x}
    \end{align*}
    \subparagraph*{(3)}
    \begin{align*}
        \lambda^2+6\lambda+9 &= 0 \\
        (\lambda +3)^2 &= 0 \\
        y &= (C_1 + C_2x)e^{-3x}
    \end{align*}
    \subparagraph*{(5)}
    \begin{align*}
        \lambda^2-\lambda+2 &= 0 \\
        \lambda &= \frac{1}{2} \pm \frac{\sqrt{7}}{2}i  \\
        y &= e^{\frac{1}{2} x}(C_1 \sin \frac{\sqrt{7}}{2}x + C_2 \cos \frac{\sqrt{7}}{2}x )
    \end{align*}
    \subparagraph*{(6)}
    \begin{align*}
        \lambda^3+2\lambda^2-\lambda &= 0 \\
        \lambda &= 0,-1\pm \sqrt{2} \\
        y &= C_1 + C_2 e^{(-1+\sqrt{2})x} + C_3 e^{(-1-\sqrt{2})x}
    \end{align*}
    \paragraph*{3}
    \subparagraph*{(1)}
    设$y = b_0 x^2 + b_1 x + b_2$
    \begin{align*}
        b_0 + -3(2b_0 x + b_1) + 5(b_0 x^2 + b_1 x + b_2) = 6 \\
        \begin{cases}
            5b_0 = 0 \\
            5b_1 -6b_0 = 0\\
            5b_2 + b_0 -3b_1 = 6
        \end{cases}
    \end{align*}

    得
    \begin{align*}
        b_0 = 0, b_1 = 0, b_2 = \frac{6}{5}\\y = \frac{6}{5}
    \end{align*}
    
    \subparagraph*{(3)}
    设$y = b_0 x^2 + b_1 x + b_2$
    \begin{align*}
        b_0 - 9(2b_0x + b_1) + 20(b_0 x^2 + b_1 x + b_2) = x + 1\\
        \begin{cases}
            20b_0 = 0 \\
            20b_1 -9b_0 = 1\\
            20b_2 - 9b_1 + b_0 = 6
        \end{cases}
    \end{align*}

    得
    \begin{align*}
        b_0 = 0, b_1 = \frac{1}{20} , b_2 = \frac{29}{400}\\y = \frac{1}{20} x + \frac{29}{400} 
    \end{align*}
    
    \end{document}