\documentclass[UTF8]{ctexart}
\usepackage{amsmath}
\title{第九周作业}
\author{邓贤杰}
\date{\today}
\begin{document}
    \maketitle
    \paragraph*{P255.2(1)}
    一致收敛,因为

    极限函数
    \begin{align*}
        f(x) = 0
    \end{align*}

    则
    \begin{align*}
        \left|\frac{1}{2^n+x^2}\right| \le \left(\frac{1}{2} \right)^n \to 0 
    \end{align*}

    \paragraph*{P255.2(2)}
    一致收敛,因为
    
    极限函数
    \begin{align*}
        f(x) = x^2
    \end{align*}

    则
    \begin{align*}
        \left|\sqrt{x^4 + e^{-n} - x^2}\right| 
        \le \left|\frac{e^{-n}}{\sqrt{x^4+e^{-n}} + x^2} \right| 
        \le e^{-n/2} \to 0 
    \end{align*}

    \paragraph*{P270.1}
    \subparagraph*{(1)}
    \begin{align*}
        \lim \limits_{n \to \infty} \frac{(n+1)2^{n+1}}{n 2^n} = \frac{1}{2} 
    \end{align*}

    故$R=2$

    \subparagraph*{(2)}
    \begin{align*}
        \lim \limits_{n \to \infty} \frac{(n+1)^k n!}{n^k (n+1)!} 
        = \lim \limits_{n \to \infty} \frac{1}{n}
        = 0 
    \end{align*}

    故$R=\infty$

    \subparagraph*{(3)}
    \begin{align*}
        \lim \limits_{n \to \infty} \frac{(n+1)!n^n}{n! (n+1)^{n+1}} 
        = \lim \limits_{n \to \infty} \left(\frac{n}{n+1}\right)^n
        = e^{-1}
    \end{align*}

    故$R=e$

    \subparagraph*{(4)}
    \begin{align*}
        \lim \limits_{n \to \infty} \frac{[(n+1)!]^2 (2n)!}{n!^2 (2n+2)!} 
        = \lim \limits_{n \to \infty} \frac{(n+1)^2}{(2n+1)(2n+2)}
        = \frac{1}{4} 
    \end{align*}

    故$R=4$

    \paragraph*{2(2)}
    \begin{align*}
        \lim \limits_{n \to \infty} \frac{na^n}{(n+1)a^{n+1}} 
        = \lim \limits_{n \to \infty} \frac{n}{(n+1)a} 
        = \frac{1}{a} 
    \end{align*}

    故$R=a$

    $x = a$时 
    \begin{align*}
        \sum_{n=1}^{\infty} \frac{1}{n}
    \end{align*}
    发散,故$a$取不到

    $x = -a$时 
    \begin{align*}
        \sum_{n=1}^{\infty} (-1)^n\frac{1}{n}
    \end{align*}
    收敛,故$-a$取到

    所以,收敛区间$(-a,a)$,收敛域$[-a, a)$

    \paragraph*{2(4)}
    \begin{align*}
        \lim \limits_{n \to \infty} \frac{(n+1)^2}{2n(n+2)} 
        = \frac{1}{2} 
    \end{align*}

    故$R=2$

    $x = 2$时 ,由
    \begin{align*}
        \frac{n}{n+1} > \frac{1}{n+1} 
    \end{align*}
    有
    \begin{align*}
        \sum_{n=1}^{\infty} \frac{n}{n+1}
    \end{align*}
    发散,故$2$取不到

    $x = -2$时 ,由
    \begin{align*}
        \frac{n}{n+1} \to 1
    \end{align*}
    有
    \begin{align*}
        \sum_{n=1}^{\infty} \frac{n}{n+1}
    \end{align*}
    发散,故$-2$取不到

    所以,收敛区间与收敛域均为$(-2,2)$

    \paragraph*{2(6)}
    \begin{align*}
        \lim \limits_{n \to \infty} \frac{3^{-n-1} + 5^{-n-1}}{3^{-n}+5^{-n}} 
        = \lim \limits_{n \to \infty} 
        \frac{\frac{1}{3}  + \frac{1}{5} \left(\frac{3}{5}\right)^{n} }{1+\left(\frac{3}{5}\right)^{n}}
        = \frac{1}{3} 
    \end{align*}

    故$R=3$

    $x = 3$时 ,由
    \begin{align*}
        1 + \left(\frac{3}{5} \right)^n \to 1
    \end{align*}
    有
    \begin{align*}
        \sum_{n=1}^{\infty} \left[1 + \left(\frac{3}{5} \right)^n\right]
    \end{align*}
    发散,故$3$取不到

    $x = -3$时 ,由
    \begin{align*}
        1 + \left(\frac{3}{5} \right)^n \to 1 \\
        -1 - \left(\frac{3}{5} \right)^n \to -1
    \end{align*}
    有
    \begin{align*}
        \sum_{n=1}^{\infty} (-1)^n \left[1 + \left(\frac{3}{5} \right)^n\right]
    \end{align*}
    发散,故$-3$取不到

    所以,收敛区间与收敛域均为$(-3,3)$
\end{document}