\documentclass[UTF8]{ctexart}
\usepackage{amsmath}
\title{第一周作业}
\author{邓贤杰}
\date{\today}

\begin{document}
    \maketitle
    \paragraph*{3}
    \begin{align*}
        &\iint \limits_D ydxdy \\
        = &\int_0^{\pi}dx\int_0^{sinx}ydy \\
        = &\int_0^{\pi}\frac{sin^2 x}{2}dx \\
        = &(\frac{x}{4}-\frac{sin2x}{8}) \big|_0^{\pi} \\
        = &\frac{\pi}{4}
    \end{align*}
    \paragraph*{4}
    \begin{align*}
        &\iint \limits_D xy^2dxdy \\
        =&\int_{-2}^{2}dy \int_{\frac{y^2}{4}}^{1} xy^2 dx \\
        =&\int_{-2}^{2} \frac{y^2}{2}-\frac{y^6}{32}dy\\
        =&\frac{32}{21}
    \end{align*}
\paragraph{5}
    \begin{align*}
        &\iint \limits_D e^{\frac{x}{y}}dxdy \\
        =&\int_0^1 dy \int_{0}^{y^2} e^{\frac{x}{y}}dx \\
        =&\int_{0}^{1} (ye^y - y)dy \\
        =& ye^y \big|_0^1 - \int_{0}^{1} e^y dy - \frac{y^2}{2} \big|_0^1\\
        =& \frac{1}{2}  
    \end{align*}
\paragraph*{6}
    \begin{align*}
        &\int_{0}^{1} dy \int_{y^{\frac{1}{3}}}^{1}\sqrt{1-x^4} dx \\
        =& \int_{0}^{1} dx \int_{0}^{x^3} \sqrt{1-x^4} dy \\
        =& -\frac{1}{4} \int_{0}^{1} \sqrt{1-x^4} \mathrm{d}(1-x^4) \\
        =& -\frac{1}{6} (1-x^4)^{\frac{3}{2}} \big|_{0}^{1} \\
        =& \frac{1}{6} 
    \end{align*}
\paragraph*{8}
\begin{align*}
    &\int_{0}^{\pi} \mathrm{d}x \int_{x}^{\pi} \frac{siny}{y} \mathrm{d}y \\
    = &\int_{0}^{\pi} \mathrm{d}y \int_{0}^{y} \frac{siny}{y} \mathrm{d}x \\
    = &\int_{0}^{\pi} siny \mathrm{d}y \\
    = &2
\end{align*}
\paragraph*{9}
\begin{align*}
    &\int_{0}^{2} \mathrm{d}x \int_{x}^{2} 2y^2sin(xy)\mathrm{d}y \\
    = &\int_{0}^{2} \mathrm{d}y \int_{0}^{y} 2y^2sin(xy)\mathrm{d}x \\
    =& \int_{0}^{2} \mathrm{d}y (-\frac{cos(xy)}{y} \big|_{0}^{y}) \\
    =& \int_{0}^{2} (1-cos(y^2))\mathrm{d}(y^2) \\
    =&4-sin4
\end{align*}
\paragraph*{10}
\begin{align*}
    & \iint \limits_D y^2 \sqrt{1-x^2} \mathrm{d}\sigma \\
    =& \int_{-1}^{1} \mathrm{d}x \int_{\sqrt{1-x^2}}^{-\sqrt{1-x^2} } y^2 \sqrt{1-x^2} \mathrm{d}y \\
    =&  \frac{2}{3} \int_{-1}^{1}  (1-x^2)^2 \mathrm{d}x \\
    =& \frac{32}{45} 
\end{align*}
\paragraph*{11}
按照$x$的正负将$D$划分为$D_1$和$D_2$
\begin{align*}
    & \iint \limits_D (|x| + y) \mathrm{d}\sigma \\
    =& \iint \limits_{D_1} (x+y) \mathrm{d}\sigma + \iint \limits_{D_2} (y-x) \mathrm{d}\sigma \\
    & \iint \limits_{D_1} (x+y) \mathrm{d}\sigma \\
    =& \int_{0}^{1} \mathrm{d}x\int_{x-1}^{1-x} (x+y)\mathrm{d}y \\
    =& \int_{0}^{1} 2x(1-x)\mathrm{d}x \\
    =& \frac{1}{3}  \\
    & \iint \limits_{D_2} (y-x) \mathrm{d}\sigma \\
    =& \int_{-1}^{0} \mathrm{d}x \int_{-x-1}^{x+1} (y-x)\mathrm{d}y \\
    =& \int_{-1}^{0} -2x(x+1)\mathrm{d}x \\
    =& \frac{1}{3} 
\end{align*}
故所求为$\frac{2}{3}$
\paragraph*{13}
\begin{align*}
    &int_{0}^{1} \mathrm{d}x \int_{\sqrt{1-x^2} }^{0} (x^2 + y^2)\mathrm{d}y \\
    =& \int_{0}^{\frac{\pi}{2} } \mathrm{d}\theta \int_{0}^{1} r^3 \mathrm{d}r \\
    =& \int_{0}^{\frac{\pi}{2} } \frac{1}{4} \mathrm{d}\theta \\
    =& \frac{\pi}{8} 
\end{align*}
\paragraph*{15}
\begin{align*}
    &\int_{0}^{2} \mathrm{d}x \int_{0}^{\sqrt{1-(x-1)^2}}3xy \mathrm{d}y \\
    =& \int_{0}^{\frac{\pi}{2} } \mathrm{d}\theta \int_{0}^{2cos\theta} 3r^3sin\theta cos\theta \mathrm{d}r \\
    =& \int_{0}^{\frac{\pi}{2} }  12cos^5\theta sin\theta \mathrm{d}\theta \\
    =& 2
\end{align*}
\paragraph*{18}
由题意可知,$\theta \in [0, \frac{\pi}{2}]\cup[\frac{3\pi}{2}, \pi]$

$D$关于极轴对称,故$\theta \in [0, \frac{\pi}{2} ]$对应的区域设为$D_1$
\begin{align*}
    &\iint \limits_D r \mathrm{d}\sigma \\
    =& 2\iint \limits_{D_1} r \mathrm{d}\sigma \\
    =& 2 \int_{0}^{\frac{\pi}{2}} \mathrm{d}\theta \int_{a}^{a(1+cos\theta)} r \mathrm{d}r \\
    =& a^2 \int_{0}^{\frac{\pi}{2}} (cos^2\theta+2cos\theta)\mathrm{d}\theta \\
    =& a^2 \int_{0}^{\frac{\pi}{2}} (\frac{1}{2} +\frac{cos2\theta}{2}+2cos\theta) \mathrm{d}\theta \\
    =& a^2 (\frac{1}{2} \theta +\frac{sin2\theta}{4} +2sin\theta)
    \big|_{0}^{\frac{\pi}{2}} \\
    =& a^2 (\frac{\pi}{4}+2) 
    % =& 2(\int_{0}^{\frac{\pi}{2} } \mathrm{d}\theta \int_{0}^{a} r \mathrm{d}r + 
    % \int_{\frac{\pi}{2} }^{\pi} \mathrm{d} \theta \int_{0}^{a(1+cos\theta)} r \mathrm{d}r) \\
    % \\
    % &\int_{0}^{\frac{\pi}{2} } \mathrm{d}\theta \int_{0}^{a} r \mathrm{d}r \\
    % =& \int_{0}^{\frac{\pi}{2} } \frac{a^2}{2} \mathrm{d}\theta \\
    % =& \frac{\pi a^2}{4} \\
    % \\
    % &\int_{\frac{\pi}{2} }^{\pi} \mathrm{d} \theta \int_{0}^{a(1+cos\theta)} r \mathrm{d}r) \\
    % =& \int_{\frac{\pi}{2} }^{\pi} \frac{a^2(1+cos\theta)^2}{2}\mathrm{d}\theta \\ 
    % =& \frac{a^2}{2}  \int_{\frac{\pi}{2} }^{\pi} (\frac{1}{2} cos2\theta + 2cos\theta + \frac{3}{2} ) \mathrm{d}\theta \\
    % =& \frac{a^2}{2} (\frac{1}{4} sin2\theta + 2sin\theta + \frac{3}{2} \theta) \big|_{\frac{\pi}{2} }^{\pi} \\
    % =& \frac{a^2}{2} (\frac{3}{4} \pi - 2)
\end{align*}
\paragraph*{19}
证:
\begin{align*}
    S=& \iint \limits_D \mathrm{d}x \mathrm{d}y \\
     =& \iint \limits_D r \mathrm{d}r \mathrm{d}\theta \\
     =& \int_{\alpha}^{\beta} \int_{0}^{r(\theta)} r \mathrm{d}r \\
     =& \frac{1}{2} \int_{\alpha}^{\beta} [r(\theta)]^2 \mathrm{d}\theta
\end{align*}
\paragraph*{20}
设所求区域为$D$,面积为$S$,可知图形关于极轴对称,故有:
\begin{align*}
    \frac{1}{2} S=& \iint \limits_D \mathrm{d}x \mathrm{d}y \\
     =& \iint \limits_D r \mathrm{d}r \mathrm{d}\theta \\
     =& \int_{0}^{\pi} \mathrm{d}\theta \int_{0}^{a(1 + cos\theta)} r \mathrm{d}r \\
     =& \frac{1}{2} \int_{0}^{\pi} a^2(1+cos\theta)^2 \mathrm{d}\theta \\
     =& \frac{a^2}{2} \int_{0}^{\pi} \frac{1}{2} cos2\theta + 2cos\theta + \frac{3}{2} \mathrm{d}\theta \\
     =& \frac{a^2}{2} (\frac{3}{2} \theta + \frac{sin2\theta}{4} + 2sin\theta) \big|_{0}^{\pi} \\
     =& \frac{3}{4} a^2 \pi
\end{align*}

故$S=\frac{3}{2} a^2 \pi$

\paragraph*{22}
令$\xi = \sqrt{xy}, \eta = \sqrt{\frac{y}{x} } $

则
\[
    \frac{D(x,y)}{D(\xi,\eta)} = 
    \left| \begin{array}{cccc}
        \frac{1}{\eta} & -\frac{\xi}{\eta^2} \\
        \eta & \xi
    \end{array} \right| = \frac{2\xi}{\eta} \\
\]

故
\begin{align*}
    &\iint \limits_D (\sqrt{\frac{y}{x}} + \sqrt{xy} )\mathrm{d}x \mathrm{d}y \\
    =& \int_{1}^{2} \mathrm{d}\eta \int_{1}^{3} \frac{2\xi}{\eta} (\xi + \eta)\mathrm{d}\xi \\
    =& \int_{1}^{2} (\frac{52}{3\eta} +8) \mathrm{d}\eta \\
    =& \frac{52}{3} ln2 + 8
\end{align*}
\paragraph*{24}
令$x=arcos\theta,y=brsin\theta$

则\[
\frac{D(x,y)}{D(r,\theta)} = \left| \begin{array}{cccc}
    acos\theta & -arsin\theta \\
    bsin\theta & brcos\theta
\end{array}\right|
= abr
\]

则
\begin{align*}
    &\iint \limits_{\Omega} (x^2 + y^2)\mathrm{d}x \mathrm{d}y\\
    =& \iint \limits_{\Omega'} ab(a^2 + b^2)r^3 \mathrm{d}r \mathrm{d}\theta\\
    =& ab(a^2+b^2)\int_{0}^{2\pi} \mathrm{d}\theta \int_{0}^{1}  r^3 \mathrm{d}r\\
    =& ab(a^2+b^2)\int_{0}^{2\pi}  \frac{1}{4} \mathrm{d}\theta\\
    =& \frac{\pi}{2}ab(a^2+b^2)  \\
\end{align*}
\end{document}