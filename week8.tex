\documentclass[UTF8]{ctexart}
\usepackage{amsmath}
\title{第八周作业}
\author{邓贤杰}
\date{\today}
\begin{document}
    \maketitle
    \paragraph*{P222.2}
    \subparagraph*{(1)}
    收敛,由比值判别法:
    \begin{align*}
        &\lim \limits_{n \to \infty} \left. \frac{(n+1)^5}{(n+1)!} \middle/ \frac{n^5}{n!} \right. \\
        =& \lim \limits_{n \to \infty} \frac{(n+1)^4}{n^5} =0 < 1
    \end{align*}
    
    \subparagraph*{(2)}
    发散,由比值判别法:
    \begin{align*}
        &\lim \limits_{n \to \infty} \left. \frac{(n+1)!}{3(n+1)^2} \middle/ \frac{n!}{3n^2} \right. \\
        =& \lim \limits_{n \to \infty} \frac{n^2}{n+1} = +\infty > 1
    \end{align*}

    \subparagraph*{(7)}
    收敛,由比值判别法:
    \begin{align*}
        &\lim \limits_{n \to \infty} \left. \frac{1000^(n+1)}{(n+1)!} \middle/ \frac{1000^n}{n!} \right. \\
        =& \lim \limits_{n \to \infty} \frac{1000}{n+1} = 0 < 1
    \end{align*}

    \subparagraph*{(9)}
    收敛,由比值判别法:
    \begin{align*}
        &\lim \limits_{n \to \infty} \left. \frac{1}{3^{n+1}} \left(\frac{n+2}{n+1}\right)^{(n+1)^2} \middle/
         \frac{1}{3^n} \left(\frac{n+1}{n}\right)^{n^2} \right. \\
        =& \lim \limits_{n \to \infty} \frac{1}{3} 
        \left[\frac{(n+1)n}{(n+1)^2} \right]^{n^2} \left(\frac{n+2}{n+1} \right)^{2n+1} \\
        =& \frac{1}{3} e^{-1} e^2 = \frac{e}{3} < 1
    \end{align*}

    \paragraph*{P234.1}
    \subparagraph*{(1)}
    绝对收敛.由于:
    \begin{align*}
        \left|(-1)^{n-1}\frac{1}{(2n)^2}\right| = \frac{1}{(2n)^2}  = \frac{1}{4n^2} 
    \end{align*}

    为收敛的p-级数

    \subparagraph*{(2)}
    当$p \le 1$时为条件收敛,$p>1$时为绝对收敛.
    
    a.$p > 1$:
    \begin{align*}
        \left|(-1)^{n+1}\frac{1}{(2n-1)^p}\right| = \frac{1}{(2n-1)^p}  < \frac{1}{2^p (n-1)^p} 
    \end{align*}
    
    为收敛的p-级数

    b.$p \le 1$,由莱布尼兹判别法:

    $\frac{1}{(2n-1)^p}$单调减
    
    且
    \begin{align*}
        \lim \limits_{n \to \infty} \frac{1}{(2n-1)^p} = 0
    \end{align*}

    故级数收敛
    
    又
    \begin{align*}
        \left|(-1)^{n+1}\frac{1}{(2n-1)^p}\right| = \frac{1}{(2n-1)^p}  > \frac{1}{2^p n^p} 
    \end{align*}

    为发散的p-级数

    故级数条件收敛

    \subparagraph*{(4)}
    条件收敛

    由莱布尼兹判别法:

    $$\frac{\sqrt{n}-1}{n}$$
    
    单调减
    
    且
    \begin{align*}
        \lim \limits_{n \to \infty} \frac{\sqrt{n}-1}{n} = 0
    \end{align*}

    故级数收敛
    
    又
    \begin{align*}
        \left|(-1)^{n}\frac{\sqrt{n}-1}{n}\right| = \frac{\sqrt{n}-1}{n} =\frac{1}{\sqrt{n}} - \frac{1}{n}  > 
        \frac{2}{n} - \frac{1}{n} = \frac{1}{n} (n>4)
    \end{align*}

    为发散的p-级数

    故级数条件收敛

    \subparagraph*{(5)}
    条件收敛

    由莱布尼兹判别法:

    $\frac{2n+1}{n(n+1)}$单调减
    
    且
    \begin{align*}
        \lim \limits_{n \to \infty} \frac{2n+1}{n(n+1)} = 0
    \end{align*}

    故级数收敛
    
    又
    \begin{align*}
        \left|(-1)^{n-1}\frac{2n+1}{n(n+1)}\right| = \frac{2n+1}{n(n+1)} = \frac{1}{n} + \frac{1}{n+1}
    \end{align*}

    为两发散的p-级数之和

    故级数条件收敛
\end{document}